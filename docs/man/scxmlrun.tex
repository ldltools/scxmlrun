\documentclass[fancy]{article}
\setDate{November 2018}
\begin{document}

\begin{Name}{1}{scxmlrun}{author}{LDL Tools}{title}
scxmlrun -- MQTT-enabled SCXML interpreter
\end{Name}

\section{Synopsis}
scxmlrun <option>* <scxmlfile>? <infile>?

\section{Description}
scxmlrun reads \emph{<scxmlfile>}, and then processes 
a sequence of input events listed in \emph{<infile>} as instructed in \emph{<scxmlfile>}.
Instead of reading from \emph{<infile>},
input events can be received over the network via MQTT.
Output events emitted using \emph{SCXML.send} in \emph{<scxmlfile>} can be
either written to a local file or transmitted via MQTT.

\section{Options}
\begin{description}
%
\item[\Opt{--model} <scxmlfile>]
  reads a state-chart in SCXML from \emph{<scxmlfile>}

% -----
\item[\Opt{-i} <infile>]
  read (lines of) SCXML events (in JSON) from \emph{<infile>}
%
\item[\Opt{-o} <outfile>]
  write output events (in JSON) to \emph{<outfile>}
%
\item[\Opt{--trace} <tracefile>]
  write trace info to \emph{<tracefile>}.
  Trace reports which states execution visits through.
%

% -----
\item[\Opt{--mqtt\Lbr:in|:out\Rbr}]
  assume MQTT-based I/O by default.\\
  \Opt{--mqtt:in}, \Opt{--mqtt:out}, or \Opt{--mqtt}
  assumes MQTT for input, output, or both, respectively.

%
\item[\Opt{--broker} <host>\Lbr:<port>\Rbr]
  specifiy \emph{<host>} (at \emph{<port>}) as the MQTT broker to connect to.\\
  (default: localhost:1883)
%
\item[\Opt{--sub} <topic>]
  subscribe to \emph{<topic>} for input events.\\
  multiple topics can be specified as: --sub t1 --sub t2 ...
%
\item[\Opt{--pub} <topic>]
  specify \emph{<topic>} as the default topic for output events.
  This is used by \emph{SCXML.send} in \emph{<scxmlfile>} when no topic is passed as a parameter.
%
\item[\Opt{--trace-pub} <topic>]
  specify \emph{<topic>} for publishing trace info

% -----
\item[\Opt{-r}, \Opt{--relay}]
  run \textbf{scxmlrun} as a MQTT \emph{repeater}, that is,
  \textbf{scxmlrun} reads input events and transmits them without any interpretation.
  No SCXML model should be specified in this mode.

% -----
\item[\Opt{-v}, \Opt{--verbose}]
  become verbose.
%
\item[\Opt{-q}, \Opt{--silent}]
  stay quiet.
%
\item[\Opt{-V}, \Opt{--version}]
  show version info.
%
\item[\Opt{-h}, \Opt{--help}]
  show usage.

\end{description}

\section{Examples}
Consider the following simple SCXML model, which we here call "echo.scxml".

\begin{verbatim}
<scxml xmlns="http://www.w3.org/2005/07/scxml" version="1.0" datamodel="ecmascript" initial="q1">
  <state id="q1">
    <transition target="q2" event="echo">
      <script>console.log (_event.data)</script>
    </transition>
  </state>
  <final id="q2"/>
</scxml>
\end{verbatim}

\begin{enumerate}
\item reading event(s) from a local file

\texttt{$ echo '\LBr"event":\LBr"name":"echo","data":"hello"\RBr\RBr' | scxmlrun echo.scxml}\\
\texttt{hello}

\item reading event(s) via MQTT

\texttt{$ scxmlrun echo.scxml --sub echo}\\
\texttt{$ mosquitto\_pub -t echo -m '\LBr"event":\LBr"name":"echo","data":"hello"\RBr\RBr'}\\
\texttt{hello}

\end{enumerate}

\section{Input/Output Event Format}

Input and output of \textbf{scxmlrun} are SCXML events in the JSON format.\\
Each event is a JSON object of the form:
\begin{center}
\LBr"event":\LBr"name":..., ...\RBr, "type":"mqtt", "topic":..., ...\RBr
\end{center}

See \URL{https://www.w3.org/TR/scxml/#send} for the detail.

Note that, as a \textbf{scxmlrun}-specific feature,
"topic"-key and its value can be included in each input/output event,
which is referred to when connecting to the MQTT broker.

\section{Environment Variables}

\emph{MQTT\_HOST} is looked up, when no broker is explicitly specified.

\section{See Also}
rules2scxml

\section{Author}
LDLTools development team at IBM Research.

\begin{itemize}
\item URL: \URL{https://ldltools.github.io}
\item Email: \Email{ldltools@outlook.com}
\end{itemize}

\section{Copyright}
(C) Copyright IBM Corp. 2018.
License Apache 2.0.\\

This is free software: you are free to change and redistribute it.
There is NO  WARRANTY,  to the extent permitted by law.

\end{document}
