\documentclass[fancy]{article}
\setDate{November 2018}
\begin{document}

\begin{Name}{1}{scxmlrun}{author}{LDL Tools}{title}
scxmlrun -- MQTT-enabled SCXML interpreter
\end{Name}

\section{Synopsis}
scxmlrun <option>* <scxmlfile>? <infile>?

\section{Description}
scxmlrun reads a state-chart in SCXML from <scxmlfile> and
runs it against input events,
which can be provided from <infile> or via MQTT.

\section{Options}
\begin{description}
%
\item[\Opt{--model} <scxmlfile>]
reads a state-chart in SCXML from <scxmlfile>

% -----
\item[\Opt{-i} <infile>]
reads lines of SCXML events (in JSON) from <infile>
%
\item[\Opt{-o} <outfile>]
write output events (in JSON) to <outfile>
%
\item[\Opt{--trace} <tracefile>]
write trace info to <tracefile>
%

% -----
\item[\Opt{--mqtt} <host>]
specifiy <host> as the MQTT broker to connect to
%
\item[\Opt{--sub} <topic>]
subscribe to <topic> for input events
%
\item[\Opt{--pub} <topic>]
specify <topic> for output events
%
\item[\Opt{--trace-pub} <topic>]
specify <topic> for publishing trace info

% -----
\item[\Opt{-v}, \Opt{--verbose}]
become verbose
%
\item[\Opt{-q}, \Opt{--silent}]
stay quiet
%
\item[\Opt{-V}, \Opt{--version}]
show version info
%
\item[\Opt{-h}, \Opt{--help}]
show usage
\end{description}

\section{See Also}
rules2scxml

\section{Author}
\URL{https://ldltools.github.io}\\
\Email{ldltools@outlook.com}

\section{Copyright}
(C) Copyright IBM Corp. 2018.

\end{document}
